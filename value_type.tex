\documentclass{article}
\usepackage[utf8]{inputenc}
\usepackage{anyfontsize}
\usepackage{amsmath}

\begin{document}
\noindent
{\large Value Representation:}
\par the DNA sequence representing a certain value begins with a type code, indicating the value type. The following is a stream of nucleotides, with fixed or volatile length. There are 4 types of nucleotide, mapped to the 4 digits of quaternary:
\begin{center}
    \begin{tabular}{c|c|c}
        nucleotide & quaternary digit & binary digits \\
        \hline
        A & 0 & 00 \\
        T & 1 & 01 \\
        C & 2 & 10 \\
        G & 3 & 11 \\
    \end{tabular}
\end{center}
The value representation is based on this mapping principle.\\[6pt]
\textbf{signed integer:}\\
Type code: A(non-negative), T(negative)\\
Length: volatile\\
Representation: 
\par A integer's type code may be \texttt{A} or \texttt{T}. \texttt{A} stands for non-nagative numbers and \texttt{T} stands for negative ones. A 2-nt sequence is attached to the type code, encoding the length of the quaternary representation of $|x|$ which is stored in the following sequence.
\par When a integer is not used for representing the value of an attribute, but the size information of some container(e.g., the length of a string or the total number of optional attributes in a tag), the type code is omitted.
\begin{center}
    \begin{tabular}{|c|c|c|}
        \hline
        A/T & length: 2 nt's & value: \textit{length} nt's \\
        \hline
    \end{tabular}
    \\[3pt]{\small \textbf{Figure 1.} DNA pattern of a signed integer.}
\end{center}
\textbf{floating-point number:}\\
Type code: C\\
Length: 16 nts\\
Representation: 
\par We store a floating-point number according to IEEE floating-point standard, with a bit sequence which will be converted into DNA sequence. The form of representation is $V=(-1)^s \times M \times 2^E$. The length of the bit sequence is 32, including a sign bit, a 8-bit exponent field and a 23-bit fraction field.(for details see IEEE 754)
%
%\par Similar to floating-point storage in computer system, we use \textit{exponent} and \textit{significand} to represent floating-point numbers with DNA sequences. A real number can be written in quaternary using scientific notation. The expression is:
%\[
%f = \textit{significand} \times 4^{\textit{exponent}}.
%\]
%The significand is stored in the first 12 nts, and the exponent is stored in the last 4 nts, with a exponent bias 127 (added to the actual exponent when encoding). For example, the quaternary number $120.123_4$ can be represented as $1.20123\times 4^2$. The encoded exponent is $2+127=(129)_{10}=(10000001)_2=(2001)_4$, thus the whole sequence will be:
%\begin{center}
%   \begin{tabular}{c}
%        0000001201232001 \\
%        \hline
%        AAAAAATCATCGCAAT
%    \end{tabular}
%\end{center}
%The digits of significand should be no more than 12, and the actual exponent ranges from -127 to 128.

\begin{center}
    \begin{tabular}{|c|c|}
        \hline
        C & value: 16 nt's \\
        \hline
    \end{tabular}
    \\[3pt]{\small \textbf{Figure 2.} DNA pattern of a floating-point number.}
\end{center}
\textbf{string:}\\
Type code: G\\
Length: volatile\\
Representation:
\par A string begins with its length(an integer indicating the number of characters in this string, and as is mentioned before, type code is omitted). We use ASCII code to encode a character. Each character in a string takes up 4 nt's (or 1 byte). 
\begin{center}
    \begin{tabular}{|c|c|c|c|}
        \hline
        G & length & content: 4*\textit{length} nt's \\
        \hline
    \end{tabular}
    \\[3pt]{\small \textbf{Figure 3.} DNA pattern of a string.}
\end{center}
\end{document}