\documentclass{article}
\usepackage[utf8]{inputenc}
\usepackage{anyfontsize}
\usepackage{amsmath}

\begin{document}
\noindent
{\large Addressing:}
\par In DNA storage, each strand in the DNA pool is a data block. To identify the position of a data block, an address sequence is attached to the strand, so that the information of the whole input string can be restored correctly. 
\par In DNA-SVG encoding, before splitting or merging the strands for appropriate sequence length, a single strand stores one tag(or element) in \texttt{.svg} file. 
\par The structure of a \texttt{.xml} file can be parsed as a tree. Each tag is a node on the tree. Thus, to identify the position of a tag in \texttt{.xml} file means to locate it on the xml tree. In tree storage, each node saves the identifier(a unique address or name) of all of its children.
For shorter sequence length, we convert the xml tree to a \textbf{left-child, right-sibling binary tree} first. After this modification, the address sequence of a tag should have the following form:
\begin{center}
    \begin{tabular}{c|c|c}
        \hline
        self identifier & left-child identifier & right-sibling identifier\\
        \hline
    \end{tabular}
    \\[3pt]{\small \textbf{Figure 1.} Address sequence.}
\end{center}
where we use the \textbf{dfs order} of the tag as its identifier.
\end{document}