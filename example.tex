\documentclass{article}
\usepackage[utf8]{inputenc}
\usepackage{DNASVG}
\usepackage{amsmath}

\title{DNA-SVG}
\author{2694824893 }
\date{September 2022}

\begin{document}



\maketitle
\noindent
{\fontsize{12pt}{14.4pt}\textbf{Decoding rules:}}
\par At both end of a DNA strand, there is primer and backward primer that are sequences representing the element type and showing where it begins and ends. The ``Required" part follows the primer, which contains every attribute that is necessary for this element. Refer to the appearing order and bps taken up, which are listed in each element's ``Required Attributes", to decode them. Example is given in Figure 1. 
\begin{center}
    \begin{tabular}{c|c|c|c|c|c}
        \hline 
        Primer & x: 4bps & y: 8bps & z: xbps & $\cdots$ & Backward Primer\\
        \hline
    \end{tabular}\\[4pt]
    Figure 1
\end{center}
\par the following is the ``Optional" part, which contains other supported attributes. At the beginning there are 4 bps implying the quantity of the following attributes. For decoding each attribute, first scan 4 for its name, then 4 bps for its length, then x bps for its content. Example is given in Figure 2.
\begin{center}
    \begin{tabular}{c|c|c|c|c|c}
        \hline 
        quantity: 4bps & $\cdots$ & name: 4bps & x: 4bps & content: xbps & $\cdots$ \\
        \hline
    \end{tabular}\\[4pt]
    Figure 2
\end{center}
\begin{Element}{rect}
    %\Primer{AGCT}
    \begin{Required}
        \Line{1}{x}{integer}
        \Line{2}{y}{floating-point}
        \Line{3}{id}{string}
    \end{Required}
    \begin{Optional}
        Global: id, class\\
        Representation: zzz, ttt
    \end{Optional}
    %\BackWardPrimer{AGCT}
\end{Element}
\begin{Element}{circle}
    %\Primer{AGCT}
    \begin{Required}
        \Line{1}{x}{integer}
        \Line{2}{y}{floating-point}
        \Line{3}{id}{string}
    \end{Required}
    \begin{Optional}
        Global: xxx, yyy\\
        Representation: zzz, ttt
    \end{Optional}
    %\BackWardPrimer{AGCT}
\end{Element}


\end{document}
